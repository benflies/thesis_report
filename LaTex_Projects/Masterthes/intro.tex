\section{Introduction}

\begin{itemize}

\item What is the subject of the study? Describe the
economic/econometric problem.

-> Implementation of Cancer Gene Panels & Variant Calling Algorithms

\item What is the purpose of the study (working hypothesis)?

-> Check whether one of them is better than the others.
-> Study effect of FFPE

\item What do we already know about the subject (literature
review)? Use citations: {\it \citet{Gallant:87} shows that...
Alternative Forms of the Wald test are considered
\citep{Breusch&Schmidt:88}.}

\item What is the innovation of the study?

-> relatively rare technique in diagnostics, but has many advantages

\item Provide an overview of your results.


\end{itemize}

General Intro

\subsection{EGFR Signaling Cascade}

The samples used in this study originate from samples that have been analyzed in
the routine workflow of the SGMB. These samples originate from patients suffering
from solid tumors, e.g. mainly melanoma, colorectal cancer (CRC) and non-small cell
lung carcinoma (NSCLC).

There is evidence that the EGFR (Epithelial Growth Factor Receptor) signaling
cascade is modified in those cancers.

Evidence suggests that many solid tumors use and modify EGFR (Epithelial
Growth Factor Receptor) signaling for their purposes [4]. Targeting this
signaling pathway is thereby an attractive anti-cancer treatment. In this
regard, anti-EGFR monoclonal antibodies (cetuximab (Erbitux®) and panitumumab
(Vectibix®) and tyrosine kinase inhibitors (erlotinib (Tarceva®) and gefitinib
(Iressa®)) have shown their usefulness in cancer treatments [5]. EGFR and
downstream proteins K-Ras / N- Ras and B-Raf are predictive biomarkers for the
successfulness of the administration of the mentioned drugs [6]. Comprehensive
information about these markers is thereby essential when choosing a suitable
treatment in order to minimize treatment- associated side-effects and to
maximize the benefits of the treatment.

In their article, Scaltriti and Baselga present the EGFR signaling pathway as a
model for targeted therapy [7]. EGFR is part of the family of receptor tyrosine
kinases. This transmembrane protein is composed of an intracytoplasmic tyrosine
kinase domain, a short hydrophobic transmembrane region and an extracellular
ligand-binding domain. Upon ligand (EGF, TGFα) binding, EGFR becomes activated.
This leads to homodimerization, which results in an auto- and
cross-phosphorylation of key tyrosine residues on its cytoplasmic domain. This
forms docking sites for cytoplasmic proteins that contain
phosphotyrosine-binding and Src homology 2 domains. This allows, amongst others,
signaling through the PTEN/PI3K/AKT and RAS-RAF-MAPK pathways. Activation of
PTEN/PI3K/AKT leads to cell growth, proliferation and survival [8], while
RAS-RAF-MAPK induces cell survival and cell cycle progression and proliferation
[9]. In the RAS-Raf-MAPK pathway, Grb2 and Sos, two adaptor proteins, form a
complex with the activated EGFR [10]. The resulting conformational change of Sos
recruits Ras-GDP, which in turn becomes activated to form Ras-GTP. Ras-GTP
activates Raf, which, in intermediate steps, phosphorylates a MAPK
(mitogen-activated protein kinase). Activated MAPKs are then imported from the
cytoplasm into the nucleus where they act on target genes. The Ras and Raf
subfamilies include several proteins, three of them are of interest in targeted
cancer treatments: K-Ras, N-Ras and B-Raf.

Activating mutations of EGFR or its downstream proteins provide resistance to
specific treatments. In that regard, the Institut National du Cancer (F) [11]
provides recommendations about which mutations have to be searched to identify
patients eligible for the administration of monoclonal antibodies or tyrosine
kinase inhibitors. For instance, mutations on codon 12 and 13 in the KRAS gene
provide resistance to the monoclonal antibody agents panitumumab and cetuximab.
Gefinitinib, a tyrosine kinase inhibitor, can only be prescribed for patients,
which show no activating mutations on EGFR.

\subsection{Targeted Sequencing and Target Enrichment Methods}
Several NGS bench-top devices have become available in the last decade. These
instrumentations differ in their underlying chemistry that influences the
instrument’s performance, accuracy, output and time per run. Common sequencing
principles include pyrosequencing (454), sequencing by ligation (SOLiD), ion
semiconductor (Ion Torrent) and sequencing by synthesis (Illumina) [12]. Even
though advances in sequencing technology and computational power and tools have
decreased the time and cost of a sequencing experiment, NGS is still mainly used
in research, with only a few laboratories using this technique in diagnostics.

In fact, validation of a NGS methodology requires careful assessment of methods
and tools [13]. Therefore, each step that is performed from the initial starting
material to sample processing, sequencing library preparation, sequencing assay
and bioinformatic processing has to be carefully checked for sources of
potential errors or variability. Basically, in the validation process, it is
checked whether the method measures what it claims to measure with the required
sensitivity and sensibility.

With the success of NGS, many cancer genomes have been sequenced and made
available to the worldwide research community. Companies, molecular diagnostics
laboratories and academic centers are trying to use these big data for their
purposes. A lot of mutations are described in these genomes, but only a small
fraction of them are clinically actionable, e.g. can be targeted with specific
drugs. Therefore, a molecular pathology laboratory does not need to perform
whole-genome or -exome sequencing, but can employ targeted NGS to analyze some
genes of interest, which include mutations for which there exists a clinical
utility. Due to the low number of target regions, targeted NGS allows high
coverage. In addition, it is a time- and cost- efficient alternative to
whole-genome or -exome sequencing. Also, targeted NGS results in a significantly
lower amount of produced data and thereby eases data storage and analysis time.
Table 1 shows a selection of commercially available cancer gene panels, which
all allow to analyze selected regions of genes implicated in cancerogenesis.
Before implementing one of these panels for molecular diagnostics, it has to be
ensured that the panel allows to study the genes of interest, e.g. genes that
are clinically applicable, and a careful assessment of its analytical validity
has to be performed.

\subsection{Illumina MiSeq Sequencing Chemistry}

\subsection{NGS Data Analysis}

GATK best practices

\subsection{Practical Implications in the Laboratory}
The quality of the genetic testing of the tumor is affected by several
factors. These include the content of tumor cells in the sample, the quality
of the tissue material, sequencing library preparation and the the
bioinformatic pipeline.

The biopsy usually consists of healthy and cancer cells. The sensitivity of
tumor variant detection is linked to the tumor cell content of the specimen. In
addition, cancers are highly heterogenous, e.g. a small subpopulation might
present mutations that provide resistance to targeted treatment. Detecting these
low-frequency mutations and clearly separating them from eventual high-frequency
fixation or sequencing artifacts presents a huge challenge [14].

Tumor biopsies usually yield a limited amount of tissue, therefore it is
conceivable to use the same sample for more analyses. In Luxembourg, all
relevant tumor biopsies are usually sent to the Laboratoire National de Santé
(LNS) to the Service of Pathologic Anatomy where the biopsy is fixed in formalin
and embedded in paraffin (FFPE). FFPE conserves the tissue morphology and
thereby allows histological analysis. In addition, it allows to store specimens
for decades. Sample quality, however, is influenced by this fixation method, but
also by the size of the biopsy, and its fixation time [13]. DNA extraction from
FFPE samples is difficult and yields low amounts of DNA [15]; formaldehyde leads
to cross-linking of nucleic acids and proteins [16]; FFPE introduces fixation
artifacts into DNA sequences [17|. These circumstances complicate sample
processing as well as NGS data interpretation. Though, FFPE samples have been
shown to be still suitable for downstream analyses [18].

FFPE = more details

Sequencing library preparation also affects the final NGS result. Several
technologies for target enrichment exist and are available for different
sequencing instruments [19]. Essential for all these enrichment methods is the
amplification of target regions and the introduction of multiplexing, which
requires the incorporation of a unique index adaptor combination for each
sample. Target enrichment methods can be separated into three basic groups:
targeted circularization, hybrid capture of target fragments and PCR-based
enrichment methods. PCR-driven methods happen on high-molecular DNA. In contrast
to uniplex long-range PCR, short-range multiplex PCR produces short DNA
fragments of target regions. There is thereby no need of DNA shearing.
Hybridization-based methods require a so-called shotgun library construction
before target regions can be captured. During this process, genomic DNA is
sheared randomly into small fragments and an adapter- and index-linked library
is produced. Biotinylated baits are added that bind to target regions. Target
regions can then be captured using streptadivin coated magnetic beads. Targeted
circularization methods rely on a digestion of DNA by restriction enzymes. The
produced DNA fragments are then circularized and uncircularized DNA fragments
are removed by exonucleases. Only circularized target regions are then amplified
by PCR.

More details

The establishment and validation of a bioinformatic NGS data analysis pipeline
still constitutes a challenge in diagnostics. After generation of FASTQ files of
the sequencer, data generally undergo quality control, followed by trimming of
low quality bases, alignment to the reference genome, variant calling and
variant annotation. For each of these steps, several bioinformatic algorithms
and tools exist [20]. The computational pipeline of the molecular pathology
laboratory has to incorporate the tools that allow the most sensitive and
sensible analysis of data. For instance, quality trimming influences the mapping
to the reference genome. The mapping, in turn, strongly affects the variant
calling. In fact, variant calling is a critical step in NGS data analysis.
Several tool kits as SAMtools, SPLINTER, VarScan2 or GATK allow variant
annotation, but vary in their false-positive and false-negative detection rates
([21], [22]). These tools have to be carefully assessed, as false-positives or
false-negatives should absolutely be avoided when it comes to the subscription
of a targeted chemotherapeutic agent.

To facilitate interpretation of NGS data, variants have to be annotated and
their clinical actionability has to be identified. Several databases have
emerged in this field (such as mycancergenome.org) and numerous tools allow to
automatize variant annotation. Here again, the choice of the database and the
variant annotator is important.

Finally, the sample-to-results time is a very pragmatic, but important factor.
The time from the biopsy to the potential start of an administration of a
targeted chemotherapeutic drug should be reduced to a minimum. For instance, in
case of late-stage cancer patients, it would be unacceptable if analysis would
take several weeks. To reduce the sample-to-results time to under two weeks, the
sample processing workflow should be as short as possible, while still yielding
high quality sequencing libraries. The bioinformatic pipeline should not only
incorporate the best tools, but should also be automatized to further reduce the
time of analysis.

\subsection{Aims of the Thesis}
Targeted NGS is still not widely used in diagnostics laboratories. The SGMB of the LNS
is planning to build expertise with the aim to adopt NGS routinely in the laboratory,
mainly in the context of diagnosis and therapy of cancer patients in Luxembourg.

The aim of this thesis project was to test commercially available cancer gene panels,
e.g. Illumina Trusight Tumor 15 and Agilent Haloplex HS ClearSeq Cancer, for their
potential use in the routine workflow of the laboratory. Several samples of cancer
patients were prepared with both kits and were sequenced on the Illumina MiSeq device.
Both kits vary in their sequencing library preparation principles: Illumina's tst15
uses the multiplex PCR approach while Agilent's Haloplex Enrichment System uses
enzymatic DNA restriction followed by probe capture.

NGS data were analyzed with the respective recommended pipelines and a custom in-house
pipeline.

Finally, several freely available variant calling algorithms were tested for their
potential implementation in the custom in-house variant discovery bioinformatic
pipeline.
